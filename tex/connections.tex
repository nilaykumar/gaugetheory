\chapter{Connections and Curvature}

\section{Basic definitions}

It is intuitive to think of a principal bundle as having ``vertical''  and ``horizontal'' components, namely the directions along the fiber and the base manifold,
respectively. The theory of connections and curvature that we detail below formalizes these notions. Though their basic idea may seem rather
innocuous, connections as we shall see are surprisingly sophisticated and have a large number of important applications. \todo{put in applications}

Before we discuss connections, however, we will define a few preliminary concepts that the reader may find familiar from Lie theory.
We shall omit the proofs of some of the statements below. For a more thorough treatment, we suggest \todo{Lee}.

\begin{defn}
    Let $G$ be a Lie group. Define the left and right-translation maps as
    \[
        \begin{array}{ccc}
            L_g(h)=gh & \text{and} & R_g(h)=hg.
        \end{array}
    \]
    Both of these maps are diffeomorphisms (multiplication and inversion are smooth) and hence the pushforwards $(L_g)_*$ and $(R_g)_*$
    acting on smooth vector fields are vector space isomorphisms. We say that a vector field $X\in\Gamma(TG)$ is \textbf{left} or \textbf{right-invariant} if
    \[
        \begin{array}{ccc}
            (L_g)_*X|_h=X|_{gh} & \text{or} & (R_g)_*X|_h=X|_{hg},
        \end{array}
    \]
    respectively, for all $g,h\in G$.
    Similarly, we say that a one-form $\omega\in\Gamma(T^*G)$ is \textbf{left} or \textbf{right-invariant} if\todo{is this correct?}
    \[
        \begin{array}{ccc}
            \omega(g)=\left( L_{hg^{-1}} \right)^*\omega(h) & \text{or} & \omega(g)=\left( R_{g^{-1}h} \right)^*\omega(h),
        \end{array}
    \]
    respectively, for all $g,h\in G$.
    %Invariance for one-forms is a little harder to visualize than for vectors fields, so let us justify
    %this definition. Evaluating $\omega$ on a vector $v\in T_gG$,
    %\begin{align*}
    %    \omega(g)(v)=\left( (L_{hg^{-1}})^*\omega(h) \right)(v)=\omega(h)\left( (L_{hg^{-1}})_*v \right)
    %\end{align*}
\end{defn}

\begin{lem}
    Let $G$ be a Lie group and $X,Y$ be left-invariant vector fields on $G$. Then their Lie bracket $[X,Y]$ is also left-invariant.
\end{lem}
\begin{proof}
    This follows straightforwardly from the naturality of the Lie bracket.\todo{refer to Lee, or do this proof}
\end{proof}

\begin{thm}
    The space of left-invariant vector fields of a Lie group $G$ forms a Lie algebra, and is in fact isomorphic (as a Lie algebra) to the Lie algebra
    $\Lie G=\fr g$ of $G$.
\end{thm}
\begin{proof}
    Omitted.\todo{make this into an exercise in the back}
\end{proof}

\todo{Adjoint and adjoint actions}

\begin{defn}
    \label{defn:conn}
    Let $P$ be a principal $G$-bundle over $M$. Let $p\in P$ be a point in the fiber above $x\in M$. Let $T_pP$ denote the tangent space to $P$ to $p$
    and denote by $V_p$ the tangent space to the fiber $\pi^{-1}(x)$ treated as a subspace of $T_pP$. We call $V_p$ the \textbf{vertical subspace} of
    $T_pP$. A \textbf{connection} $\Gamma$ on $P$ is an assignment of a \textbf{horizontal subspace} $H_p$ to each $p\in P$ such that
    \begin{enumerate}[(i)]
        \item there is a direct sum decomposition $T_pP=V_p\oplus H_p;$
        \item $H_p$ depends smoothly on $p$, as defined below.
        \item $H_{pg}=(R_g)_*H_p$ for every $g\in G$;
    \end{enumerate}
    A tangent vector $X\in T_pP$ is called \textbf{vertical} (resp. \textbf{horizontal}) if it lies in $V_p$ (resp. $H_p$). By 
    $(i)$ we see that any vector can uniquely be decomposed into vertical and horizontal components: $X=X_v+X_h$ with
    $X_v\in V_p$ and $X_h\in H_p$. This definition extends naturally to the idea of vertical and horizontal vector fields.
    Indeed, by saying that $H_p$ depends smoothly on $p$ in $(ii)$, we mean that for any smooth vector field $X\in \Gamma(TP)$, the vertical
    and horizontal components $X_v$ and $X_h$ are smooth as well. Finally, $(iii)$ requires that the choice of horizontal subspace is equivariant along the fiber direction.
\end{defn}

The above definition of a connection is reasonable, as given a vector space without an inner product defined on it, there is no canonical choice
of a horizontal subspace given an arbitrary vertical subspace. If, by chance, we are given a Riemannian metric $\eta$ on $P$, we may choose the horizontal
subspaces according to the orthogonal complement given by $\eta$. Hence the choice of metric induces a choice of connection.
We will return to this case later, for now working in full generality.

Unfortunately, thinking of a connection as a choice of horizontal subspaces,
while intuitive, is rather difficult to compute with. Hence we provide an alternate definition in terms of Lie algebra-valued one-forms, and we
prove that the two definitions are equivalent. Before we do so, let us familiarize ourselves a little bit with vector-valued forms.

\begin{defn}
    Let $V$ be a $d$-dimensional vector space and $M$ a smooth manifold. Denote by $E=M\times V$ the trivial bundle over $M$ and by $\Lambda^pT^*M$ the
    $p$th exterior power of the cotangent bundle of $M$.
    We define a \textbf{$V$-valued $p$-form} on $M$ to be a section of the tensor product bundle of $\Lambda^pT^*M$ and $E$.
    More explicitly, fixing a basis $\left\{ e_1,\ldots,e_d \right\}$ for $V$ (equivalently a global frame for $E$) we can write any $V$-valued $p$ form as:
    \[\mu_x(v_1,\ldots,v_p)=\mu_x^1(v_1,\ldots,v_p)e_1+\ldots+\mu_x^d(v_1,\ldots,v_p)e_d\]
    for all $x\in M$ and $v_i\in T_xM$, where the $\mu^i\in\Gamma(\Lambda^pT^*M)$ are the \textbf{components} of $\mu$. We say that a $V$-valued
    form is smooth if and only if its components are smooth in the usual sense of forms (we leave it as an exercise to the reader to show
    that this definition is independent of basis).\todo{exercise}
    We denote the space of smooth such sections on $M$ by $\Omega^p(M;V).$

    For $p=0$, we define by convention $\Omega^0(M;V)$ to be the space of smooth vector-valued functions on $M$.
    For $p=1$, we get smooth $V$-valued one-forms: if $\mu$ is such a form then for $x\in M$, $\mu_x$
    is a linear transformation from $T_xM$ to $V$, i.e. $\mu_x\in\Hom(T_xM;V).$ In this sense, ordinary one-forms on $M$
    may be seen as $\R$-valued one-forms.
\end{defn}

Just as with usual forms, we may pullback vector-valued forms.
\begin{defn}
    Let $M,N$ be smooth manifolds and $f:M\to N$ a smooth map. If $\mu$ is a $V$-valued $p$-form on $N$ we define the \textbf{pullback of $\mu$ by $f$}
    to be the $V$-valued $p$-form $f^*\mu$ on $M$ defined by
    \[(f^*\mu)_x(v_1,\ldots,v_p)=\mu_{f(x)}(f_*v_1,\ldots,f_*v_p)\]
    with $v_i\in T_xM$, for all $x\in M$. Note that if $\mu=\mu^ie_i$ then\todo{prove this} $f^*\mu=(f^*\mu^i)e_i$; hence if $\mu$ is smooth, so is $f^*\mu$.
\end{defn}

We now introduce the Maurer-Cartan form, a Lie algebra-valued form on a Lie group, which will be useful for redefining connections on principal bundles.
\begin{defn}
    The \textbf{Maurer-Cartan form} $\mu$ on a Lie group $G$ is the unique Lie algebra-valued one-form $\mu \in \Omega^1(G; \fr g)$
    that is left-invariant and acts as the identity map on $T_eG$. Uniqueness is clear: if two such forms exist they agree, by definiton, at
    the identity; subsequently, left-invariance forces the two forms to agree everywhere. We can in fact write out the action of $\mu$
    completely explicitly using just the definition. For some $v\in T_gG$,
    \[\mu(g)(v)=\left( (L_{g^{-1}})^*\mu(e) \right)(v)=(L_{g^{-1}})_{*g}(v).\]
\end{defn}

The following lemma relates a given basis of the Lie algebra to the Maurer-Cartan form.
\begin{lem}
    \label{LMC}
    Fix a basis $\{e^i\}$ for $\fr g=T_eG$ and hence a dual basis $\left\{ \mu^i(e) \right\}$ for $\fr g^*=T_e^*G$.
    By construction, $\mu^i(e)(e_j)=\delta^i_j$. If we denote by $\mu^i$ the left-invariant vector fields generated by $\mu^i(e)$, then 
    the Maurer-Cartan form is written
    \[\mu(g)=\sum_i\mu^i(g)e_i.\]
\end{lem}
\begin{proof}
    Let us start with the right-hand side of the identity and show that it evaluates everywhere to the left-hand side.
    Take any $v\in T_gG$ and denote by $w$ the pushforward $(L_{g^{-1}})_*v$. Then
    \begin{align*}
        \left(\sum_i\mu^ie_i\right)(v)&=\sum_i\mu^i(g)(v)e_i=\sum_i\left((L_{g^{-1}})^*\mu^i(e)\right)(v)e_i\\
        &=\sum_i \mu^i(e)(w)e_i=\sum_i\mu^i(e)( \sum_j w^j e_j )\\
        &=\sum_{i,j}w_j\mu^i(e)(e_j)e_i=\sum_iw^ie_i=w\\
        &=(L_{g^{-1}})_{*}v=\mu(g)(v),
    \end{align*}
    as desired.
\end{proof}

\begin{exmp}
    We now follow [Naber] in computing the Maurer-Cartan form $\mu$ for $G=\GL_n\R$ as an example. 

    We denote by $x^{ij}$ the coordinate functions of $\R^{n^2}$ such that $x^{ij}(g)=g^{ij}$ for $g\in\GL_n\R$. Recall that $\fr{gl}_n\R$ is the space of
    all $n\times n$ matrices. Fix a basis for $\fr {gl}_n\R$ to be $\left\{ \partial/\partial x^{ij}|_{e} \right\}_{i,j=1,\ldots n}$.
    The corresponding dual basis is $\left\{ dx^{ij}(e) \right\}_{i,j=1,\ldots n}$; to find (the components of) $\mu$, we wish to
    find left-invariant $\R$-valued one-forms $\mu^{ij}$ satisfying $\mu^{ij}(e)=dx^{ij}(e)$. Recall that left-invariance of $\mu^{ij}$ requires that
    $\mu^{ij}(g)=\left( L_{g^{-1}} \right)^*(\mu^{ij}(e))$. Let $v\in T_g\GL_n\R$ be the tangent vector to a curve $\gamma:(-\varepsilon,\varepsilon)\to\GL_n\R$,
    i.e. $\gamma'(0)=v$. Then we have
    \begin{align*}
        \mu^{ij}(g)(v)&=\mu^{ij}(e)\left( (L_{g^{-1}})_*v \right)=dx^{ij}(e)\left( (L_{g^{-1}})_*v \right)\\
        &=dx^{ij}(e)\left( (L_{g^{-1}}\circ\gamma)'(0) \right)=dx^{ij}(e)\left(\frac{d}{dt}\bigg|_{t=0}(g^{-1}\gamma(t))  \right)\\
        &=\sum_k\left(g^{-1}\right)^{ik}v^{kj}
    \end{align*}
    Rewriting this as
    \begin{align*}
        \mu^{ij}(g)(v)=\sum_k\left(g^{-1}\right)^{ik}v^{kj}=\sum_k(g^{-1})^{ik}dx^{kj}(g)(v),
    \end{align*}
    we find that
    \[\mu^{ij}(g)=\sum_k (g^{-1})^{ik}dx^{kj}(g).\]
    Using Lemma \ref{LMC} we conclude
    \[\mu(g)=\sum_{i,j=1}^n\mu^{ij}(g)\frac{\partial}{\partial x^{ij}}\bigg|_{e}=\sum_{i,j,k=1}^nx^{ik}(g^{-1})dx^{kj}(g)\frac{\partial}{\partial x^{ij}}\bigg|_{e}\]
    This is often rather slickly abbreviated as $\mu(g)=g^{-1}dg$. Let us now apply this to the simple case of $\GL_2\R$. Writing
    \[g=\begin{pmatrix}
            a&b\\c&d
        \end{pmatrix}\text{ and }g^{-1}=\frac{1}{ad-bc}\begin{pmatrix}
            d&-b\\-c&a
        \end{pmatrix},
    \]
    we can express the Maurer-Cartan form as
    \begin{align*}
        \mu(g)&=g^{-1}dg=\frac{1}{ad-bc}\begin{pmatrix}
            d&-b\\-c&a
        \end{pmatrix}\begin{pmatrix}
            dx^{11}&dx^{12}\\dx^{21}&dx^{22}
        \end{pmatrix}\\
        &=\frac{1}{ad-bc}\begin{pmatrix}
            d\;dx^{11}-b\;dx^{21}&d\;dx^{12}-b\;dx^{22}\\-c\;dx^{11}+a\;dx^{21}&-c\;dx^{12}+a\;dx^{22}.
        \end{pmatrix}
    \end{align*}
\end{exmp}

\begin{defn}
    Let $G$ be a Lie group and $\fr g$ be its Lie algebra. We say that a $\fr g$-valued one-form $\mu$ is right-equivariant if
    \[(R_g)^*\mu=\Ad(g^{-1})\circ \mu\]
    for all $g\in G$. More generally, let $P$ be a manifold equipped with smooth right action $\sigma(p,g)=p\cdot g$ of $G$ and let $\mu$ be a $\fr g$-valued one-form
    on $P$. Then we say that $\mu$ is \textbf{right-equivariant under $\sigma$} if
    \[(\sigma_g)^*\mu=\Ad(g^{-1})\circ \mu\]
    for all $g\in G$.\todo{change notation to $R_g$}
\end{defn}

\begin{lem}
    The Maurer-Cartan form is right-equivariant.
\end{lem}
\begin{proof}
    Take $g,h\in G$ and a vector $v\in T_{hg^{-1}}G$. It suffices to show right-equivariance for this case:
    \begin{align*}
        \mu(h)\left( (R_g)_*v \right)&=(L_{h^{-1}})_*\left( (R_g)_*v \right)\\
        &=(L_{g^{-1}}\circ L_{gh^{-1}})_*\left( (R_g)_*v \right)\\
        &=(L_{g^{-1}})_*\left( (L_{gh^{-1}}\circ R_g)_*v \right)\\
        &=(L_{g^{-1}})_*\left( (R_g\circ L_{gh^{-1}})_*v\right)\\
        &=\Ad(g^{-1})\circ (L_{gh^{-1}})_*v\\
        &=\Ad(g^{-1})\circ \mu(hg^{-1})(v),
    \end{align*}
    as desired. Note that we have used the covariance of the pushforward together with the fact that the left and right actions
    commute.
\end{proof}

Let us finally return to connections.
\begin{defn}
    Given a connection $\Gamma$ on a principal $G$-bundle $P$ over $M$, we define the corresponding \textbf{connection one-form} $\omega\in \Omega^1(P;\fr g)$
    as follows. Recall that the elements of $\fr g$ are in one-to-one correspondence with left-invariant vector fields on $G$. Given a Lie algebra
    element $X$, denote by $\tilde X$ its corresponding left-invariant vector field. Note that when $\tilde X$ is considered
    as a vector field on $P$ it is a purely vertical vector field. Thus, for some vector $Y\in T_pP$ with vertical component $Y_v$ we define
    $\omega(p)(Y)$ to be the (unique)\todo{why?} Lie algebra element $X$ such that $\tilde X(p)=Y_v$.\todo{smoothness?}
\end{defn}

The next theorem shows us that a connection one-form is in fact enough information to recover the connection, as desired.

\begin{thm}
Let $P$ be a principal $G$-bundle over $M$ with a connection $\Gamma$ and associated connection form $\omega$. Then the connection form satisfies:
\begin{enumerate}[(i)]
\item $\omega(\tilde X)=X$ for all $X\in\fr g$;
\item $(R_g)^*\omega=\Ad(g^{-1})\circ \omega$, i.e. $\omega$ is right-equivariant.
\end{enumerate}

Conversely, given a $\fr g$-valued one-form $\omega$ on $P$ satisfying $(i)$ and $(ii)$,
there is a unique connection $\Gamma$ on $P$ whose connection form is $\omega$.
\end{thm}
\begin{proof}
\todo{Understand this}The first property follows straight from the definition of the connection form above.
We show the second property in cases; in particular, we show that the left and right-hand sides are equal when evaluated on a vector field $X$ that is either horizontal or vertical (as defined by $\Gamma$).
In the case where $X=X_h$ is horizontal, we note that the right-hand side is zero simply by definition of the connection form.
On the left-hand side we also obtain zero, as $((R_g)^*\omega)(X)$ must be horizontal as well, by condition (iii) in the definition of $\Gamma$.
Now suppose $X=X_v$ is vertical with respect to $\Gamma$. Then $X=\tilde Y$ for some $Y\in\fr g$, i.e. $X$ is the fundamental vector field associated to $Y$.
Then $(R_g)_*X$ is the fundamental vector field corresponding to $\Ad(g^{-1})Y$.
 Hence we see that
\[(R_g^*\omega)_p(X)=\omega_{pg}((R_g)_*X)=\Ad(g^{-1})Y=\Ad(g^{-1})(\omega_p(X)),\]
as desired.

Conversely, given a form $\omega$ satisfying these two properties, we define
\[H_p=\{X\in T_pP\mid \omega(X)=0\}.\]
It now suffices to show that this choice of horizontal subspace for each $p\in P$ satisfies conditions (i)-(iii) of the connection.
It is clear that by defining $H_p$ as above we obtain a direct sum decomposition $T_pP=V_p\oplus H_p$ (as $\omega(p)$ maps onto $\fr g$). Right equivariance of the subspaces follows easily: take $X\in H_p$. Then
\begin{align*}
\omega(p\cdot g)((R_g)_*X)&=\omega((R_g)_*X)\\
&=\Ad(g^{-1})\circ\omega(X)\\
&=0,
\end{align*}
so $(R_g)_*X\in H_{p\cdot g}$. But by the injectivity of $(R_g)_*$ and the fact that $\dim H_p=\dim H_{p\cdot g}$, we find that $H_{p\cdot g}=(R_g)_*H_p$.
To show that this choice of horizontal subspaces varies smoothly over $P$ we must show that for any smooth vector field $X$ on $P$, the horizontal component vector field $X_h$ is smooth as well. But note that we can write, at any point $p\in P$:
\begin{align*}
X_h|_p&=X|_p-X_v|_p\\
&=X|_p-\widetilde{\omega_p(X)}|_p.
\end{align*}
Here we have found the vertical component $X_v$ by using the definition of the connection one-form.
Of course, $\omega$ is smoot so the map $X\mapsto \omega_p(X)$ is smooth, and since the tilde operation is an isomorphism, the map $X\mapsto \widetilde{\omega_p(X)}$ is smooth as well, and we are done.
\end{proof}






\section{The BPST connection: a first example}
Now that we introduced connections, we work through a particularly important connection 1-form on the quaternionic Hopf bundle, $Sp(1) \to S^7 \to S^4$. We will see later that this connection will give the gauge fields associated to the physical notion of isotopic spin. A family of solutions to the Yang-Mills equations for this problem is the BPST instanton solutions, hence the name ``BPST connection'' for the connection we are describing in this section. In order to perform this computation, we regard $S^7$ as unit length elements of $\HH^2$, and we denote the inclusion map by $i : S^7 \to \HH^2$. We start by defining an $\Imag \HH$-valued 1-form $\tilde \omega$ on $\HH^2$:
\[      \tilde \omega = \Imag(\bar q^1 dq^1 + \bar q^2 dq^2)     \]
Then define $\omega$ on $S^7$ by $\omega = i^* \tilde \omega$. This apparently arbitrary definition will be motivated shortly. After we work out the explicit action of this 1-form on tangent vectors, we will show that the horizontal subspaces that it generates are the orthogonal complements of $V_p S^7$ with respect to the usual quaternionic inner product. Therefore $\omega$ is a particularly nice candidate for a connection 1-form on $S^7$, as its horizontal subspaces are easy to visualize. With this motivation in mind, we work from the definition of $\omega$ as a pullback. For every $p \in S^7$ and every $X \in T_p S^7$, we have:
\[     \omega_p (X) = \tilde \omega_p (i_{*p} X)    \]
In order to unpack this notation, we write $p = (p^1, p^2) \in S^7 \subset \HH^2$, and $X = (X^1, X^2) \in T_p(S^7) \subset T_p(\HH^2) = T_{p^1} (\HH) \times T_{p^2} (\HH)$. Then:
\begin{align*}
\omega_p(X) &= \Imag(\bar q^1 dq^1 + \bar q^2 dq^2)(p^1, p^2)(X^1, X^2)  \\
&= \Imag[\bar p^1 dq^1(X^1, X^2) + \bar p^2 dq^2(X^1, X^2)] 
\end{align*}
But note that:
\[   dq^j(X^1, X^2) = (X^1(q^j), X^2(q^j)) = x^1 \frac{\p q^j}{\p q^1} + x^2 \frac{\p q^j}{\p q^2} = x^1 \delta_{j1} + x^2 \delta_{j2} \]
On the left side of the second equal sign, $X^i$ are tangent vectors, that can act on the smooth function $q^j$ and return another smooth function. On the right side of the second equal sign, $x^i$ are the components of $X^i$ in the $\frac{\p}{\p q^i}$ direction. Obviously, each $X^i$ only has a component in the $\frac{\p}{\p q^i}$ direction, because it belongs to the tangent space $T_{p^i}\HH$. Then our expression for the action of $\omega$ simplifies to:
\[       \omega_p(X) = \Imag(\bar p^1 x^1 + \bar p^2 x^2)  \]
From here we can prove that $\omega$ has the properties of a connection on $S^7$. Before we do this, though, we prove the claim made earlier that the horizontal subspaces given by $\omega$ are orthogonal complements to $V_p S^7$. For this we need compute the orthogonal complement, and show that it is equal to the kernel of $\omega$, i.e. the vectors $X \in T_p S^7$ such that $\Imag(\bar p^1 x^1 + \bar p^2 x^2) = 0$. Recall that the usual inner product on $\HH^2$ is $\langle  (p^1, p^2) , (X^1, X^2) \rangle = \Real(\bar p^1 X^1 + \bar p^2 X^2)$. We now consider $V_p S^7$, which is the same as the subspace of $T_p S^7$ that is also tangent to the fiber of $\mathcal{P}$ containing $p$. Since the fiber is $\{ (p^1 g, p^2 g) : g\in Sp(1) \}$, we can write $V_p S^7$ as $\{ (p^1 a, p^2 a) : a \in \mathfrak{sp}(1) = \Imag \HH \}$. Then the orthogonal complement to $V_p S^7$ is the set of $(v^1, v^2) \in T_p S^7$ such that:
\begin{align*}
0 &= \langle (p^1 a, p^2 a), (v^1, v^2) \rangle \\
&= \Real(\overline{p^1 a} v^1 + \overline{p^2 a} v^2) \\
&= \Real[\bar a (\bar p^1 v^1 + \bar p^2 v^2)]
\end{align*}
But $\bar a$ is purely imaginary, so this means $\Imag(\bar p^1 v^1 + \bar p^2 v^2) = 0$. But this is precisely the kernel of $\omega$.

Having established this, we can finally show that $\omega$ has the properties of a connection 1-form on $S^7$. First we show that it is right equivariant under the standard action $\sigma$ of $Sp(1)$ on $S^7$. (By the standard action we mean $\sigma_g(q^1, q^2) = (q^1 g, q^2 g)$.) In other words, we need to show that for each $g \in Sp(1), p \in S^7, X \in T_{p \cdot g^{-1}}(S^7)$, the following holds:
\[     \omega_p\big((\sigma_g)_{*p\cdot g^{-1}}(X)\big) = \Ad_{g^{-1}}\big(\omega_{p\cdot g^{-1}} (v)\big)   \]
Let's first work with the LHS. Let $\gamma_1(t), \gamma_2(t)$ be two curves in $\HH$ such that $\dot \gamma_1|_{t=0} = X^1$ and $\dot \gamma_2|_{t=0} = X^2$. Then:
\begin{align*}      
(\sigma_g)_{*p\cdot g^{-1}}(X) &= (\sigma_g)_{*p\cdot g^{-1}}(X^1, X^2) \\
&= \left. \frac{\p}{\p t} \right|_{t=0} \sigma_g (\gamma_1(t), \gamma_2(t)) \\
&= \left. \frac{\p}{\p t} \right|_{t=0} (\gamma_1(t) g, \gamma_2(t) g) \\
&= (X^1 g, X^2 g)
\end{align*}
Under the standard identification of $T_p \HH$ with $\HH$, we rewrite this as $(x^1 g, x^2g)$, where $x^1, x^2$ are the components of $X^1, X^2$. Therefore the LHS becomes $\omega_p(x^1 g, x^2g) = \Imag(\bar p^1 x^1 g + \bar p^2 x^2 g)$. On the other hand we have on the RHS:
\begin{align*}
\omega_{p\cdot g^{-1}} (X) &= \Imag(\overline{p^1 g^{-1}}x^1 + \overline{p^2 g^{-1}} x^2) \\
&= \Imag(\overline{g^{-1}} \bar p^1 x^1 + \overline{g^{-1}} \bar p^2 x^2) \\
&= \Imag(g \bar p^1 x^1 + g \bar p^2 x^2)
\end{align*}
Since $g\in Sp(1)$ implies $g = \overline{g^{-1}}$. Then we have:
\[     RHS = g^{-1} \Imag(g \bar p^1 x^1 + g \bar p^2 x^2) g  = \Imag( \bar p^1 x^1 g +  \bar p^2 x^2 g)     \]
Therefore $LHS = RHS$, and this shows that $\omega$ is right equivariant.

The other property we need to check is that $\omega$ acts trivially on fundamental vector fields, i.e. $\omega_p (\tilde A (p)) = A$. This is a simple consequence of the definition of $\tilde A$ as the velocity vector at $t=0$ of the curve:
\[ \gamma_p(t) = p \cdot \exp(tA) =  \big(p^1 \exp(tA) , p^2 \exp(tA)\big) \]
Explicitly computing the velocity vector gives $\tilde A(p) = (p^1 A, p^2 A)$. Then:
\[    \omega_p(\tilde A(p)) = \Imag( \bar p^1 p^1 A + \bar p^2 p^2 A ) = \Imag((|p^1|^2 + |p^2|^2 )A) = \Imag(A) = A   \]
This finishes the proof that $\omega$ is a connection on the bundle $Sp(1) \to S^7 \to S^4$. In later sections we will come back to this principal bundle and the BPST connection $\omega$, and use them to study the isotopic spin of a physical system.

We briefly mention here that an analogous construction exists for the complex Hopf bundle, $U(1) \to S^3 \to S^2$. Here we regard $S^3$ as unit length vectors in $\C^2$, we let $i:S^3 \to \C^2$ be the inclusion map and we define a 1-form on $S^3$ as:
\[       \omega = i^* (\bar z^1 dz^1 + \bar z^2 dz^2)       \]
One can then show that this 1-form is a connection, and that it is the natural connection for $S^3$, in the sense that its horizontal subspaces are orthogonal complements in $\C^2$ to $V_p S^3$. We leave these calculations as exercise. We note, however, the beautiful similarity between the connections for the quaternionic and complex Hopf bundles. Modulo some issues of noncommutativity for quaternions, one can go from one to the other by simply exchanging the quaternions $q$ for complex numbers $z$ and viceversa.

We will also work with the complex Hopf bundle and its natural connection in the later sections. Concretely, we will use it to describle the electrodynamics of a magnetic monopole.


















\section{Curvature}

Now that we are somewhat familiar with connections on principal bundles, the next object
of interest will be the curvature associated to a connection. At this point in time we lack
the background to justify the choice of name `curvature,' but we shall forge ahead regardless. 
Before we define the curvature, however, it will be useful to state some generalities concerning
exterior derivatives and wedge products of vector-valued forms. Note that when not stated otherwise,
the contents of this section will be in the context of a principal $G$-bundle $P$ over $M$.

\begin{defn}
Let $\omega$ be a $\fr g$-valued $p$-form on $\omega$ a manifold $P$ and $X_i$ be vector fields on $P$.
We define the \textbf{exterior derivative of} $\omega$ to be the $(p+1)$-form given in coordinates as
\[d\omega = (d\omega^i)e_i\]
where $\{e_i\}$ is a basis for $\fr g$, and $d$ is the usual exterior derivative.
\end{defn}

\begin{defn}
    \label{defn:wedge}
    Let $\omega$ and $\eta$ be $V$-valued $p$- and $q$-forms, respectively. We can define a $V$-valued $(p+q)$-form
    called the \textbf{wedge product} $[\omega\wedge\eta]$ as
    \[ [\omega\wedge\eta](v,w)=\frac{1}{p!q!}\sum_{\sigma\in S_{p+q}}\sgn(\sigma)\big[\omega(\sigma\cdot v),\eta(\sigma\cdot w)\big],\]
    where $[-,-]$ is a bilinear map $V\times V\to V$, $S_{p+q}$ is the symmetric group on $p+q$ letters,
    and $v,w$ are vectors with $p$ and $q$ components each (with the action of $S_{p+q}$ being the usual permutation on components).
    Note the similarity to the usual definition of wedge product except for the bilinear map $[-,-]$, which
    is required as there is no product operation on vectors in general. We have kept the notation similar to
    that of the commutator as we will primarily be working with Lie algebra valued forms.
\end{defn}

\begin{rmk}
    Unlike for the usual wedge product, where $\omega\wedge\omega=0$ for a one-form $\omega$, the wedge product of a
    $V$-valued one-form $\eta$ with itself is not necessarily zero:
    \[ [\eta\wedge\eta](v,w)=[\eta(v),\eta(w)]-[\eta(w),\eta(v)=2[\eta(v),\eta(w)].\]
\end{rmk}

\begin{defn}
We define the \textbf{curvature} $\Omega$ of a connection one-form $\omega$ as follows. For each $p\in P$ (for $P$ a principal $G$-bundle over $M$) and for all $X,Y\in T_pP$, we let $\Omega$ be the $\fr g$-valued two-form given by
\[\Omega_p(X,Y)=(d\omega)_p(X_h,Y_h),\]
i.e. the exterior derivative of the connection evaluated on the horizontal components of its arguments. The curvature form $\Omega$
is indeed smooth: $d\omega$ is smooth since the connection is smooth, and $X_h,Y_h$ are smooth if $X,Y$ (as vector fields) are smooth
by \hyperref[defn:conn]{Definition \ref*{defn:conn}}. Given two vector fields $X,Y$ on $P$, we can compute
\begin{align*}
    \Omega(X,Y)&=d\omega(X_h,Y_h)\\
    &=X_h\omega(Y_h)-Y_h\omega(X_h)-\omega([X_h,Y_h])\\
    &=-\omega([X_h,Y_h]),
\end{align*}
where we have used the usual formula for the exterior derivative of a one-form (convince yourself that this holds in the $\fr g$-valued
case as well, with the vector fields acting componentwise\todo{make this an exercise?}) as well as the properties of the connection. This gives us a geometric flavor
of what the curvature does; in particular, it returns zero if and only if $[X_h,Y_h]$ is horizontal.
\end{defn}

The following results expose some of the useful properties of $\Omega$.

\begin{lem}
Let $P$ be a principal $G$-bundle over $M$ with $\omega$ a connection one-form. Then the curvature $\Omega$ of $\omega$ is right-equivariant:
\[(R_g)^*\Omega(X,Y)=\Ad(g^{-1})\circ \Omega(X,Y).\]
\end{lem}
\begin{proof}
As the exterior derivative commutes with pullbacks by smooth maps, we have that
\begin{align*}
(R_g)^*\Omega(X,Y)&=R_g^*(d\omega)(X_h,Y_h)\\
&=d((R_g)^*\omega)(X_h,Y_h)\\
&=d(\Ad(g^{-1})\omega)(X_h,Y_h)\\
&=\Ad(g^{-1})d\omega(X_h,Y_h)\\
&=\Ad(g^{-1})\Omega(X,Y)
\end{align*}
\end{proof}

\begin{lem}[Structure equation]
    \label{lem:curvature_structure}
    The curvature $\Omega$ of a connection one-form $\omega$ satisfies 
    \begin{equation}
        \Omega=d\omega + \frac{1}{2}[\omega\wedge\omega].
        \label{eq:curvature_structure}
    \end{equation}
\end{lem}
\begin{proof}
    Using the \hyperref[defn:wedge]{definition} of the wedge product, it suffices to show that
    \[d\omega(X_h,Y_h)=d\omega(X,Y)+[\omega(X),\omega(Y)],\]
    for all smooth vector fields $X,Y$ on $P$. Writing $X=X_h+X_v$ and $Y=Y_h+Y_v$ and expanding
    via bilinearity, one finds that it suffices to prove the structure equation for the three cases:
    $X,Y$ both horizontal, $X,Y$ both vertical, and $X,Y$ vertical and horiztonal respectively.
    The first case is easy: $\omega(X)=\omega(X_h)=0$ and $\omega(Y)=\omega(Y_h)=0$. The commutator vanishes
    and we are left with $d\omega(X_h,Y_h)=d\omega(X,Y)$, which is true in this case. For $X,Y$ both
    vertical, we note that $X$ and $Y$ can be written in terms of fundamental vector fields, $X=\tilde A,Y=\tilde B$
    for some $A,B\in\fr g$. In this case, using the fact that $\omega(\tilde A)=A$ and similarly for $B$, we must have that
    \begin{align*}
        0&=d\omega(\tilde A,\tilde B)+[\omega(\tilde A),\omega(\tilde B)]\\
        &=\tilde AB-\tilde BA-\omega\left([\tilde A,\tilde B]\right)+[A,B]\\
        &=-\omega\left(\widetilde{[A,B]}\right)+[A,B]\\
        &=-[A,B]+[A,B]=0,
    \end{align*}
    where we have used the fact that $A$ and $B$ are constant $\fr g$-valued one-forms, thus vanishing under
    derivation by vector fields. Finally, in the case of $X=\tilde A$ vertical and $Y$ horizontal, we find that
    \begin{align*}
        0&=d\omega(\tilde A,Y_h)\\
        &=\tilde A\omega(Y_h)-Y_hA-\omega([\tilde A,Y_h])\\
        &=-\omega([\tilde A,Y_h]).
    \end{align*}
    It remains to show that the Lie bracket of a vertical vector field with a horizontal vector field is again
    horizontal. To do this, let $\alpha$ be the flow of $\tilde A$; then the Lie
    bracket is the Lie derivative\todo{Refer to Lee for Lie derivatives}
    \begin{align*}
        [\tilde A, Y_h]_p=\lim_{t\to 0}\frac{d(\alpha_{-t})_{\alpha_t(p)}(Y_h)_{\alpha_t(p)}-(Y_h)_p}{t}.
    \end{align*}
    We can write the integral curve determined by $A$ as $\alpha_t(p)=p\cdot e^{tA}$ and thus
    \[d(\alpha_{-t})_{\alpha_t(p)}=d(R_{e^{-tA}})_{p\cdot e^{tA}}.\]
    By the \hyperref[defn:conn]{right-equivariance} of horizontal subspaces, since $(Y_h)_{p\cdot e^{tA}}$ is horizontal,
    so is $d(R_{e^{-tA}})_{p\cdot e^{tA}}(Y_h)_{p\cdot e^{tA}}$ as a vector in $T_pP$; consequently, the Lie derivative $[\tilde A,Y_h]$ is a
    horizontal vector field. This concludes the verification of the structure equation.
\end{proof}




\section{Local descriptions: gauge fields}
\todo{cite Jose}
So far we have been focusing on describing the total space of the bundle, and have not said much about the base manifold $M$. However, in physical applications we are interested in integrating over $M$ (``real'', physical space, such as the spacetime manifold $\R^4$) quantities which descend from the total space $P$. In fact, the so-called gauge fields used in quantum field theory are just pullbacks of the connection form to $M$. Therefore we take some time now to develop a local formalism on the base manifold; it's local because gauge fields are defined in terms of a cannonical section $s_{\alpha}$, which in turn depends on the local trivialization $\psi_{\alpha}$ we are working with. With this in mind, we make the following definition:

\begin{defn}
Consider a local trivialization $\psi_{\alpha}: \pi^{-1}(U_{\alpha}) \to U_{\alpha} \times G$ and the associated canonical section $\sigma_{\alpha} : U_{\alpha} \to \pi^{-1}(U_{\alpha})$. A \textbf{gauge field} is the following $\fr g$-valued 1-form on $U_{\alpha}$:
\[      A_{\alpha} = s_{\alpha}^* \omega  \in \Omega^1(U_{\alpha}, \fr g)   \]
\end{defn}

Since the gauge field is local, it's natural to ask what the relation is between fields on different trivializations, $A_{\alpha}$ and $A_{\beta}$. To obtain this relation, we will use the following lemma, that gives a local description of the global form $\omega$.

\begin{lem}
The restriction of the connection form $\omega$ to a local trivialization $\pi^{-1}(U_{\alpha})$ agrees with the following 1-form:
\[    \omega_{\alpha} = \Ad_{g_{\alpha}^{-1}} \circ \pi^* A_{\alpha} + g_{\alpha}^* \mu     \]
$\mu$ is the Maurer-Cartan form, while $g_{\alpha}$ is the fiber component of the local trivialization: $\psi_{\alpha}(p) = (\pi(p), g_{\alpha}(p))$.
\end{lem}

\begin{proof}
We prove this lemma in two parts. In the first, we show that $\omega$ and $\omega_{\alpha}$ agree on the image of $s_{\alpha}$, and in the second, we show that the forms transform the same way under the right action of $G$.
\begin{enumerate}
\item Let $m \in U_{\alpha}$ and $p = s_{\alpha}(m)$. We note that any vector $v \in T_pP$ can be decomposed into its horizontal and vertical parts:
\[   v = (s_{\alpha})_* \pi_* (v) + \bar v  \]
Then we apply $\omega_{\alpha}$ to $v$ split in this way, and use the fact that $g_{\alpha}$ evaluates to the identity on the image of $s_{\alpha}$. This means we can ignore the adjoint action of $g_{\alpha}^{-1}$ in the definition of $\omega_{\alpha}$, and we get:
\begin{align*}
\omega_{\alpha}(v) &= (\pi^* s_{\alpha}^* \omega)(v) + (g_{\alpha}^* \mu_e) (v) \\
&= \omega((s_{\alpha})_* \pi_* v) + \mu_e((g_{\alpha})_* v) \\
&= \omega((s_{\alpha})_* \pi_* v) + \mu_e((g_{\alpha})_* \bar v) \\
&= \omega((s_{\alpha})_* \pi_* v) + \omega(\bar v) \\
&= \omega((s_{\alpha})_* \pi_* v + \bar v) \\
&= \omega(v)
\end{align*}
On the second line we used the equivariance of $\theta$, and on the fourth the fact that the connection form acts as the Maurer-Cartan on vertical vectors.
\item Now we check the behavior of $\omega_{\alpha}$ under $R_g^*$:
\begin{align*}
R_g^*(\omega_{\alpha})_{pg} &= \Ad_{g_{\alpha}(pg)^{-1}} \circ R_g^* \pi^* s_{\alpha}^* \omega + R_{g}^* g_{\alpha}^* \mu \\
&= \Ad_{(g_{\alpha}(p)g)^{-1}} \circ R_g^* \pi^* s_{\alpha}^* \omega +  g_{\alpha}^* R_{g}^* \mu \\
&=  \Ad_{g^{-1}(g_{\alpha}(p))^{-1}} \circ  \pi^* s_{\alpha}^* \omega +  g_{\alpha}^*( \Ad_{g^{-1}}\circ \mu) \\
&= \Ad_{g^{-1}} ( \Ad_{(g_{\alpha}(p))^{-1}} \circ  \pi^* s_{\alpha}^* \omega +  g_{\alpha}^*\mu  ) \\
&= \Ad_{g^{-1}} \circ (\omega_{\alpha})_p
\end{align*}
Which shows that $\omega$ and $\omega_{\alpha}$ behave the same way under the right action of $G$. Note that, on the third line, we used the fact that $\pi \circ R_g = \pi$.
\end{enumerate}

\end{proof}


\section{References}

For coverage of the basics of Lie theory (such as left-invariance, etc.) we recommend \cite{Lee}.
Our treatment of connections is based primarily on \cite{KN}, with explicit examples/computations
taken from \cite{Naber1}. The sections on curvature, gauge fields and gauge transformations follow
the general structure of \cite{jose}, but are hopefully clearer.

\section{Exercises}


