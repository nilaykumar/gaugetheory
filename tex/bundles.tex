\chapter{Principal Bundles}

\section{Definition and examples}\todo{add more basic content}

Let $G$ be a Lie group with identity $e$. A \textbf{right $G$-space} is a smooth manifold $X$ equipped with a smooth right action $X\times G\to X$. 
A \textbf{left $G$-space} is defined similarly, though we will primarily deal with right $G$-spaces in these notes\todo{add footnote about conventions, see Naber}.
We will typically denote the action of $G$ on a $G$-space by a ``$\cdot$'', even when there may
be multiple $G$-spaces present, as the particular action should be clear from the context.
If $X$ and $Y$ are two right $G$-spaces, a map $\phi:X\to Y$ is said to be \textbf{$G$-equivariant} (or equivalently, a map of right $G$-spaces) if the diagram
\[\begin{tikzcd}
        X\arrow[swap]{d}{g}\arrow{r}{\phi}&Y\arrow{d}{g}\\
        X\arrow{r}{\phi}&Y
\end{tikzcd}\]
commutes, i.e. $\phi(x\cdot g)=\phi(x)\cdot g$ in the case of left $G$-spaces.

\begin{defn}
    Let $X$ and $P$ be right $G$-spaces and $\pi: P\to X$ be a smooth $G$-equivariant projection onto $X$.
    We say that $(P,\pi)$ is a \textbf{principal $G$-bundle over the base space} $X$ if the following two conditions hold:
    \begin{enumerate}[(i)]
        \item $G$ acts trivially on $X$, i.e. $x\cdot g=x$ for all $x\in X$ and $g\in G$;
        \item $X$ has an open cover $\{U_\alpha\}_{\alpha\in A}$ for which there exist $G$-equivariant diffeomorphisms $\phi_\alpha:\pi^{-1}(U_\alpha)\to U_\alpha\times G$ called \textbf{local trivializations} such that the diagram
            \[\begin{tikzcd}[column sep=small]
                \pi^{-1}(U_\alpha)\arrow[swap]{dr}{\pi}\arrow{rr}{\phi_\alpha}&&U\times G\arrow{dl}{\pi_\alpha}\\
                &U_\alpha&
            \end{tikzcd}\]
            commutes, where $\pi_\alpha$ is projection onto the first component. Here, $U\times G$ has the structure of a right $G$-space such that $(u,g)\cdot h=(u,gh)$.
    \end{enumerate}
    We will also often denote the bundle $(P,\pi)$ as $G\to P\to X$.
\end{defn}

Note that the first condition above is equivalent to requiring the action of $G$ to preserve the fibers of $\pi$,
\[\pi(p\cdot g)=\pi(p)\cdot g=\pi(p).\]
In fact, we can say more, as the following lemma shows.

\begin{lem}
    Given an $x\in X$ and some $p\in\pi^{-1}(x)$ above $x$, the orbit of $p$ under the action of $G$ is the whole fiber:
    \[\pi^{-1}(x)=\{p\cdot g: g\in G\}=p\cdot G.\]
\end{lem}
\begin{proof}
    The inclusion $p\cdot G\subseteq \pi^{-1}(x)$ holds due to condition (i). To show the reverse inclusion, $\pi^{-1}(x)\subseteq p\cdot G$, it suffices to show that for any $q\in\pi^{-1}(x)$, there exists an $h\in G$ such that $p\cdot h=q$. Using condition (ii) we can take advantage of the equivariant local trivialization and work instead in local coordinates, $\phi(p)=(x, g_1)$ and $\phi(q)=(x,g_2)$, for some $g_1,g_2\in G$. Then $\phi(p)\cdot (g_1^{-1}g_2)=\phi(q)$, which implies (by injectivity of $\phi$) that $p\cdot h=q$ for $h=g_1^{-1}g_2$.
\end{proof}

Hence we may identify the fibers of $\pi$ with the group $G$.
Informally, then, one can think of a principal bundle as a space that is locally just the product of a base manifold $X$ with a Lie group $G$.
Globally, however, there may exist topological complications; consider, for example, the (infinite) M\"obius strip $M$.
Locally $M$ is just $S^1\times \R$, but globally there is some kind of un-productlike twisting. Those familiar with vector bundles
may have found the M\"obius strip a good example to keep in mind - in fact, as we shall show shortly, $M$ can also be
treated as a principal bundle. 

Before we move on to some examples of principal bundles, let us prove the following useful lemma.

\begin{lem}
    $G$ acts freely on $P$.
\end{lem}
\begin{proof}
    Let us compute the stabilizer of some $p\in P$ where $\phi(p)=(x,g)$.
    In coordinates, the requirement that $h\in\Stab_p$ is
    $\phi(p\cdot h)=(x,gh)=(x,g)=\phi(p).$
    This, of course, implies that $h=e$.
\end{proof}

To be more explicit, let us now consider some examples, a few of which will become important later on.\todo{be careful about smooth actions}\todo{add more examples}

\begin{exmp}
    The obvious example of a principal bundle is a bundle that is globally just a product. In other words, given
    a Lie group $G$ and a smooth manifold $X$, we can construct the \textbf{trivial $G$-bundle over $X$} to be
    $G\to X\times G\overset{\pi}{\to} X$ where $\pi$ is simply the projection onto the first factor and $X\times G$ is treated as
    a $G$-space under the action $(x,g)\cdot h=(x,gh)$. 
\end{exmp}

\begin{exmp}
    The reader familiar with projective spaces will recall that the sphere $S^n$ is a double cover of the real projective space $\RP^n$.
    Thus we may consider the natural action of $O(1)=\Z_2=\left\{ 1,-1 \right\}$ (given the discrete topology) on $S^n$ identifying antipodal points,
    i.e. for any $p\in S^n$,
    \[p\cdot (\pm 1)=(x^1,\ldots, x^{n+1})\cdot (\pm 1)=\pm (x^1,\ldots,x^{n+1}).\]
    We can thus construct the principal bundle $O(1)\to S^{n+1}\overset{\pi}{\to} \RP^{n+1}$. Let us check that the required properties are satisfied.
    First note that $O(1)$ does indeed act trivially on the base space, as multiplication by a scalar preserves, by construction, points in $\RP^{n+1}$.
    Next let us construct local trivializations, i.e. $O(1)$-equivariant diffeomorphisms $\phi_\alpha:\pi^{-1}(U_\alpha)\to U_\alpha\times G$ where
    $U_\alpha$ for $\alpha\in\{1,\ldots, n+1\}$ are the usual charts on $\RP^n$ given by the non-vanishing of the $\alpha$th homogeneous coordinate.
    If we now consider the graphical charts for the sphere given by 
    \begin{align*}
        V_\beta^+&=\left\{ (x^1,\ldots,x^{n+1})\in S^n\mid x^\beta>0 \right\}\\
        V_\beta^-&=\left\{ (x^1,\ldots,x^{n+1})\in S^n\mid x^\beta<0 \right\},
    \end{align*}
    we see that
    \[\pi^{-1}(U_\alpha)=V_\alpha^+\cup V_\alpha^-.\]
    Hence we define the map $\phi_\alpha:\pi^{-1}(U_\alpha)\to U_\alpha\times O(1)$ as
    \[\phi_\alpha\left( (x^1,\ldots,x^{n+1}) \right)=\left( [x^1:\ldots:x^{n+1}], \sgn x^{\alpha} \right).\]
    Note first that $\phi_\alpha$ is indeed equivariant,
    \[\phi_\alpha(x\cdot \pm 1)=\phi_\alpha(\pm x)=([x],\sgn \pm x^\alpha)=([x],\pm \sgn x^\alpha)=\phi_\alpha(x)\cdot (\pm 1),\]
    and smooth as the first component is simply a smooth projection and the second component is constant and hence smooth in the disjoint $V_\beta^+$
    and $V_\beta^-$ each. Moreover, $\phi_\alpha$ has an inverse given simply by treating a point in projective space as a point in the sphere
    and multiplying by the given sign. This is certainly smooth - hence $\phi_\alpha$ is indeed a local trivializtion, and $S^n$ forms a principal
    $O(1)$-bundle over $\RP^n$.

    We leave it as an exercise at the end of the chapter to construct similar bundles over the complex and quaternionic projective spaces $\CP^n$ and $\HH\Proj^n$.
\end{exmp}

\begin{exmp}
    Let $G$ be a Lie group, with $H$ a closed subgroup. Let us show that $G$ is a principal $H$-bundle over the coset space $G/H$.
    Let us treat $G$ and $G/H$ as right $H$-spaces with the obvious right-multiplication. Then the action of $H$ on $G/H$ is trivial:
    \[gH\cdot h = gH\]
    for all $g\in G$ and $h\in H$. Showing local triviality is a little trickier. The interested reader may find a proof in \todo{Brocker-Dieck, thm 4.3}.
\end{exmp}

\begin{exmp}
    Let $E\to X$ be a real vector bundle of rank $k$ over a smooth $n$-dimensional manifold $X$.
    Define a \textbf{frame} at a point $x\in X$, to be an ordered basis $f=(f_1,\ldots, f_k)$ for the fiber $E_x$ above $x$. Denote the set of such frames by $F_x$.
    Consider now the set $F(E)=\coprod_{x\in X} F_x$, known as the \textbf{frame bundle associated to $E$}. We claim that $F(E)$ is a principal
    $\GL_k$-bundle over $X$, with the projection map given by $\pi(f)=x$ for $f\in F_x$.

    Leaving aside issues concerning the smooth structure of $F(E)$ for a moment, let us instead begin by defining the action of $\GL_k$ on $F(E)$:
    intuitively, the set of bases of $E_x$ comes equipped with a natural action of $\GL_k$.
    Hence, for $f\in F(E), g\in\GL_k$, we define via the usual matrix multiplication,
    \[(f\cdot g)_i=\sum_{j=1}^kg_{ij}f_j.\]
    Note that this action takes
    frames at $x$ to frames at $x$ since $\pi(f\cdot g)=\pi(f)$, thus rendering the action trivial on $X$ (and making $\pi$ equivariant). Next, in order to construct
    local trivializations for $F(E)$, (again modulo smoothness) we must construct bijections $\phi_\alpha:\pi^{-1}(U_\alpha)\to U_\alpha\times\GL_k$,
    where $\{U_\alpha\}$ is an open cover of $X$. Fixing a frame $e_x\in E_x$ at each $x\in U_\alpha$, we define, for $f_x\in F_x=\pi^{-1}(x)$
    \[\phi_\alpha(f_x)=(x, g_x),\]
    where $g_x\in\GL_k$ satisfies $f=e_x\cdot g_x$. In other words, we've fixed a `reference' or `identity' frame in each fiber, and are parameterizing
    frames $f$ by the element in $\GL_k$ that takes the reference frame to $f$ (and $x$, of course). Thus defined, $\phi_\alpha$ commutes with the
    necessary projection maps and is a bijection (this follows by linear algebra: given one frame at $x$, one can reach any other frame at $x$ uniquely. Indeed,
    the action of $\GL_k$ on $F_x$ is free and transitive, though the lack of a preferred frame renders the bijection non-canonical).
    Furthermore, $\phi_\alpha$ is equivariant with respect to the natural right $\GL_k$-action defined on $U_\alpha\times\GL_k$.

    It now remains to equip $F(E)$ with a smooth structure such that $\pi$ is smooth, the action of $\GL_k$ is smooth, and $\phi_\alpha$ are
    diffeomorphisms. To do this, we use as charts $U_\alpha\times V_i$ where $U_\alpha$ are the charts of $X$ and $V_i$ are the usual $k^2$
    charts associated to $\GL_k$. We then have coordinates $\psi:F(E)\to\R^{n+k^2}$ given by $\psi(p)=(x_i,g_{ij})$, where $x=\pi(p)$ and $g$
    is defined as in the trivialization above. Then $\pi$ is a smooth projection and the right action of $\GL_k$ is smooth. Finally, it is easily
    verified that both $\phi_\alpha$ and $\phi_\alpha^{-1}$ are smooth; this concludes the proof that $F(E)$ is a principal $\GL_k$-bundle.
    
    %To see why $F_x$ should have the structure of $\GL_k$, note first that a frame $f=(e_1,\ldots,e_k)\in F_x$
    %can be viewed as a linear isomorphism $f_x:\R^k\to E_x$ given by $f(a^1,\ldots, a^k)=a^1e_1+\cdots a^ke_k$. Then $F_x$ is naturally
    %equipped with a right-action of $\GL_k$ given by composition: $f\cdot g=f\circ g:\R^k\to E_x$. As is well-known from linear algebra,
    %there is a unique linear transformation sending one basis to any other, and hence this right-action is both free and transitive.
    %
    %To see this, note first that $\GL_k\R$ acts transitively 
    %and freely on the set of frames at $x$ (the us) on each $F_x$, giving $F_x$ the structure of $\GL_k$.

    %%choosing a frame for $E_x$ is equivalent to choosing a linear isomorphism from $\R^k$ to $\pi^{-1}(x)$, i.e. choosing an
    %element of $\GL(k)$ that takes the usual orthonormal basis of $\R^k$ to the chosen frame.
    %Of course, $F_x$ is equipped with a natural right action by
    %$\GL(k)$ (given by right-multiplication), which is free and transitive. In this sense, the set of frames $F_x$ is, when viewed as a Lie group, diffeomorphic to $\GL(k)$.
    %\todo{finish frame bundles, see DuPont}
\end{exmp}

\begin{rmk}
    In the
    case of $E$ a Riemannian vector bundle (e.g. the tangent bundle of a Riemannian manifold) we may also consider the frame bundles
    $OF(E)$ and $SOF(E)$ as the orthogonal and oriented orthogonal frame bundles, respectively.
\end{rmk}

Now that we have worked through a few concrete examples of principal bundles, it should be clear that non-trivial bundles are precisely that: non-trivial.
With this in mind, one might ask exactly how a
given bundle fails to be trivial. This can be done via \textbf{transition functions}, in analogy to those used in defining smooth manifolds.
In particular, we investigate how the local coordinates of a point in a principal bundle depend on the choice of local trivialization (assuming, of
course, that the point is in a region of overlap $U_\alpha\cap U_\beta$).

\begin{defn}
    Let $P$ be a principal $G$-bundle over $X$ with local trivializations $\Psi_\alpha$ defined over an open cover $X=\{U_\alpha\}$.
    Fix $\alpha,\beta$ such that $U_\alpha\cap U_\beta$ is non-empty. 
    On the overlap, consider
    \[\Psi_\beta\circ\Psi_\alpha^{-1}:(U_\alpha\cap U_\beta)\times G\to (U_\alpha\cap U_\beta)\times G,\]
    which takes, say,
    \[(\Psi_\beta\circ\Psi_\alpha^{-1})(x,g)=(x,h)\]
    for some $x\in X$ and $g,h\in G$. As $\Psi_\beta\circ\Psi_\alpha$ is a diffeomorphism, $h$ must depend smoothly on $g$,
    and hence we may write
    \[(\Psi_\beta\circ\Psi_\alpha^{-1})(x,g)=(x,g_{\beta\alpha}(x)h),\]
    where $g_{\beta\alpha}$ is a smooth map $U_\alpha\cap U_\beta\to G$ called the \textbf{transition function} $g_{\beta\alpha}$. 
    It should be clear that these transition functions satisfy the so-called \textbf{cocycle condition}:
    \[g_{\alpha\beta}(x)\cdot g_{\beta\gamma}(x)=g_{\alpha\gamma}(x),\]
    for all $\alpha,\beta,\gamma$, with $x\in U_\alpha\cap U_\beta\cap U_\gamma$.
\end{defn}

Though perhaps not evident at first, these transition functions (defined over some open cover of the base manifold) hold enough
information to recreate the principal bundle in its entirety. Intuitively, this is because the bundle can be thought of
as a set of local products $U_\alpha\times G$ that is glued together in a non-trivial way via the transition functions in order to
form the global structure of the bundle. We leave this as an exercise that the reader is urged to carry out (see \hyperref[exc:bundleconst]{Exercise \ref*{exc:bundleconst}}).
\todo{maybe do this out}

\section{Morphisms and sections}\todo{expand on this section}

We may define morphisms between princpal $G$-bundles, thus defining the category of principal $G$-bundles.
\begin{defn}
    Let $G\overset{\pi_1}{\to}P_1\to X_1$ and $G\overset{\pi_2}{\to}P_2\to X_2$ be two principal $G$-bundles. We define a \textbf{bundle
    map} between $P_1$ and $P_2$ to be a smooth $G$-equivariant map $\Phi:P_1\to P_2$:
    \[\Phi(p\cdot g)=\Phi(p)\cdot g,\]
    for all $p\in P_1, g\in G$, with the group action on both $P_1,P_2$ denoted by `$\cdot$'. Note that bundle maps preserve fibers.
\end{defn}

We will be concerned primarily with maps between bundles with the same base space. This is quite a special case, as the following theorem shows.
\begin{thm}
    A bundle map $f$ between two principal $G$-bundles $P_1$ and $P_2$ over the same base $X$ is an isomorphism.
\end{thm}
\begin{proof}
    Let us first suppose that $P_1,P_2$ are both trivial, i.e. $P_1=P_2=X\times G$. Then, since $f$ must preserve fibers, we see that
    \[f(x,g)=(x,\sigma(x)g)\]
    for some function $\sigma:X\to G$.
    \todo{finish}
\end{proof}
Indeed, this result suggests how strong the condition of being a principal bundle is.


% Bundle maps

Just as with vector bundle, we can construct ``sections'' of a principal bundle, i.e. globally twisted functions on our base manifold.

\begin{defn}
    Let $G\to P\overset{\pi}{\to} X$ be a principal $G$-bundle and $U$ be an open neighborhood of $X$. A \textbf{local section} (sometimes \textbf{cross-section}) of $\pi$ is a continuous map $\sigma: U\to P$ such that $\pi(\sigma(x))$. If $U=X$, we say that $\sigma$ is a \textbf{global section}.
\end{defn}

\begin{exmp}
    Given a local trivialization $\Psi:\pi^{-1}(U)\to U\times G$, we can define a local section $\sigma:U\to P$ as $\sigma(x)=\Psi^{-1}(x,e)$. In words, we simply take the identity section in our local trivialization (that assigns to each point $x$ on the manifold the identity element in the copy of $G$ above $x$) and then pull it back onto the bundle. As the section is continuous in the local trivialization and because $\Psi$ is continuous and takes fiber to fiber, we see that $\sigma$ is indeed a local section. We will refer to this section as the \textbf{canonical section} associated to the trivialization $\Psi$.
\end{exmp}

\begin{thm}
    A principal bundle $G\to P\overset{\pi}{\to} X$ is trivial iff it has a global section.
\end{thm}
\begin{proof}
    Let $\sigma:X\to P$ be a global section. Consider now the trivial bundle $G\to X\times G\overset{\rho}{\to}X$; it suffices to find a bundle map $\Theta: X\times G\to P$ making the diagram 
    \begin{equation*}
        \begin{tikzcd}
            X\times G\arrow{rr}{\Theta}\arrow[swap]{rd}{\rho}&&P\arrow[bend left]{ld}{\pi}\\
            &X\arrow[bend left]{ru}{\sigma}&
        \end{tikzcd}
    \end{equation*}
    commute. Define the map $\Theta:X\times G\to P$ by
    \[\Theta(x,g)=\sigma(x)\cdot g.\]
    This map clearly takes fiber to fiber and is $G$-equivariant, as
    \[\Theta(x,gh)=\sigma(x)\cdot (gh)=(\sigma(x)\cdot g)\cdot g.\]
    Hence $\Theta$ defines a bundle equivalence between $X\times G$ and $P$, and we are done. Intuitively, one thinks of the given global
    section $\sigma$ as a (non-canonical) choice of a reference ``identity'' section.

    Conversely, suppose $P\cong X\times G$. We can easily construct the global identity section $\sigma: X\to X\times G$ given by $\sigma(x)=(x,e)$.
\end{proof}

\section{Pullback bundles}

\section{Associated bundles}

Throughout this section, we fix a principal bundle $G\to P\xrightarrow{\pi}X$, as well as a 
left $G$-space $F$. \todo{motivation?}\todo{cite Naber}

\begin{defn}
Consider the product manifold $P\times F$ as a right $G$-space, with a joint action defined as follows:
\[(p,v)\cdot g\equiv(p\cdot g, \rho(g^{-1})\cdot y),\]
for $p\in P,y\in V$, and $g\in G$. We define the \textbf{fiber bundle associated to $P$ with fiber $F$} to be the orbit space
\[P\times_G F\equiv (P\times F)/G,\]
i.e. the quotient of the product $P\times F$ by the action defined above. The following lemma
justifies the name.
\end{defn}

\begin{lem}
    The fiber bundle associated to $P$ as above forms a smooth, locally trivial bundle with base space $X$ and fiber $F$
    written as $F\to P\times_GF\xrightarrow{\mu}X$. This structure is unique\todo{why?}.
\end{lem}
\begin{proof}
    Concretely, we may write the points of the quotient $P\times_G F$ as equivalence classes $[p,y]$ under the relation
    \[[p,y]=[p\cdot g,g^{-1}\cdot y].\]
    If we denote by $q:P\times F\to P\times_G F$ the quotient map given by
    \[q(p,y)=[p,y],\]
    we can give $P\times_G F$ the induced quotient topology. Furthermore, we can define $\mu:P\times_G F\to X$ by
    \[\mu([p,y])=[\pi(p)],\]
    which is well-defined: $\mu([p\cdot g,g^{-1}\cdot y])=\pi(p\cdot g)=\pi(p)$. Finally, defining
    $\pi_P:P\times F\to P$ to be the obvious projection, we find that the diagram
    \begin{equation*}
        \begin{tikzcd}
            P\times F\arrow{r}{q}\arrow{d}{\pi_P}&P\times_\rho F\arrow{d}{\mu}\\
            P\arrow{r}{\pi}&X
        \end{tikzcd}
    \end{equation*}
    commutes, implying that $\mu$ is in fact a continuous map (characteristic property of quotients).
    
    Let us now construct the necessary local trivializations for $P\times_G F$.
    Fix an open set $U$ in $X$. Associated to $U$ we have a local trivialization of $P$
    given by $\Psi:\pi^{-1}(U)\to U\times G$:
    \[\Psi(p)=(\pi(p),\psi(p)),\]
    where $\psi(p\cdot g)=\psi(p)g$ is the projection onto the group component for all $p\in\pi^{-1}(U)$ and $g\in G$.
    If we denote by $s:U\to\pi^{-1}(U)$ the canonical section $s(x)=\Psi^{-1}(x,e)$, we can construct a map
    $\tilde\Phi:U\times F\to\mu^{-1}(U)$ by
    \[\tilde\Phi(x,y)=[s(x),y].\]
    For $\tilde\Phi$ to be a local trivialization it must first be a homeomorphism. It is clearly continuous - it
    suffices to show that it is bijective with continuous inverse.
    Note first that for any $x\in X$ with $p\in\pi^{-1}(x)$,
    the preimage $\pi_P^{-1}(p)$ is sent by $q$ to elements of the form $[p,y]$ for some $y\in F$ and hence
    $\mu^{-1}(x)=\left\{ [p,y]\mid y\in F \right\}$. 
    Thus we can write
    \begin{align*}
        \mu^{-1}(U)&=\bigcup_{x\in U}\mu^{-1}(x)\\
        &=\bigcup_{x\in U}\left\{ [p,y]\mid \pi(p)=x,y\in F \right\}\\
        &=q(\pi^{-1}(U)\times F).
    \end{align*}
    We first show that $\tilde\Phi$ is surjective. Fix $[p,y]\in \mu^{-1}(U)$ for $p\in P, y\in F$
    and let $\pi(p)=x=\pi(s(x))$. As there exists some $g\in G$ for which $p=s(x)\cdot g$ it follows that
    \[\tilde\Phi(x,g\cdot y)=[s(x),g\cdot y]=[s(x)\cdot g,(g^{-1}g)\cdot y]=[p,y].\]
    Next, we show injectivity. Suppose $\tilde\Phi(x,y)=\tilde\Phi(x',y')$. Then, by definition of $\tilde\Phi$,
    we must have that $[s(x),y]=[s(x'),y']$, and in particular, that $\pi(s(x))=\pi(s(x'))\implies x=x'$.
    Consequently, $[s(x),y]=[s(x),y']$, i.e. there exists some $g\in G$ such that $s(x)\cdot g=s(x)$
    and $y'=g^{-1}\cdot y$. The action of $G$ on a fiber of $P$ is free, and hence $g=e$, implying that
    $y=y'$. Thus $\tilde\Phi$ is bijective with inverse is given by
    \[\tilde\Psi([p,y])=(\pi(p), g\cdot y),\]
    where $p=s(\pi(p))\cdot g$ (check this!). Using the characteristic property of quotient maps, $\tilde\Psi$
    is continuous because the map $\tilde\Psi\circ q:q^{-1}(\mu^{-1}(U))=\pi^{-1}(U)\times F\to U\times F$ given
    by $(p,y)\to(\pi(p),g\cdot y)$ is continuous.

    Note that $\mu(\tilde\Phi(x,y))=\mu([s(x),y])=\pi(s(x))=x$ and hence $\tilde\Phi$ is (the inverse of) a local trivialization
    for the bundle $F\to P\times_GF\xrightarrow{\mu}X$ (we leave it as an exercise to show that $P\times_GF$
    is Hausdorff\todo{exercise: Naber 1.3.23}). It now suffices to construct a differentiable structure on $P\times_GF$
    in respect to which $\mu$ is smooth and the maps $\tilde\Psi_\alpha:\mu^{-1}(U_\alpha)\to U_\alpha\times F$
    are diffeomorphisms for an open cover $X=\left\{ U_\alpha \right\}$. Consider first two local trivializations
    $\Psi_\alpha,\Psi_\beta$ (with $U_\alpha\cap U_\beta$ non-empty) for $P$ with let $s_\alpha,s_\beta$ the
    corresponding canonical local sections and $g_{\beta\alpha}:U_\alpha\cap U_\beta\to G$ the transition function.
    By the above procedure, we obtain the homeomorphisms $\tilde\Psi_\alpha:\mu^{-1}(U_\alpha)\to U_\alpha\times F$
    and $\tilde\Psi_\beta:\mu^{-1}(U_\beta)\to U_\beta\times F$. Consider now the map
    \[\tilde\Psi_\beta\circ\tilde\Psi_\alpha^{-1}:(U_\alpha\cap U_\beta)\times F\to(U_\alpha\cap U_\beta)\times F,\]
    which, for $(x,y)\in(U_\alpha\cap U_\beta)\times F$, takes
    \begin{align*}
        \tilde\Psi_\beta\circ\tilde\Psi_\alpha^{-1}(x,y)&=\tilde\Psi_\beta\left( \tilde\Phi_\alpha(x,y) \right)\\
        &=\tilde\Psi_\beta([s_\alpha(x),y])=\tilde\Psi_\beta([s_\beta(x)\cdot g_{\beta\alpha}(x),y])\\
        &=\tilde\Psi_\beta([s_\beta(x)\cdot g_{\beta\alpha}(x),g_{\beta\alpha}^{-1}(x)\cdot(g_{\beta\alpha}(x)\cdot y)])\\
        &=\tilde\Psi_\beta([s_\beta(x),g_{\beta\alpha}(x)\cdot y])\\
        &=(x,g_{\beta\alpha}(x),y).
    \end{align*}
    This computation shows that $\tilde\Psi_\beta\circ\tilde\Psi_\alpha^{-1}$ is in fact a diffeomorphism, as it
    is smooth and clearly has a smooth inverse.
    %Note additionally that since $\left\{ U_\alpha \right\}$ is an open
    %cover for $X$, $\left\{ \mu^{-1}(U_\alpha) \right\}$ is an open cover for $P\times_GF$. We can now define a smooth
    %structure on $P\times_GF$ by taking $(\mu^{-1}(U_\alpha),\tilde\Psi_\alpha)$ as the charts (which are smoothly
    %compatible by the above calculation).
    \todo{finish; see Lee lem 10.6}
\end{proof}

\section{References}

There are a number of excellent references on the subject of principal bundles, including \cite{KN}, \cite{Spivak},
and \cite{Naber1}. The discussion of associated bundles follows that of \cite{Naber1}.

\section{Exercises}

\begin{exc}
    \label{exc:bundleconst}
    Let $X$ be a smooth manifold and let $\left\{ U_\alpha \right\}$ be an open cover of $X$. Suppose for each
    $\alpha,\beta$ we have smooth smooth transition functions $g_{\alpha\beta}:U_\alpha\cap U\beta\to G$ that
    satisfy the cocycle condition. Show that there is a smooth principal $G$-bundle $P\xrightarrow{\pi} X$ with smooth local
    trivializations $\Psi_\alpha:\pi^{-1}(U_\alpha)\to U_\alpha\times G$ whose transition functions are the given maps
    $g_{\alpha\beta}$. (Hint: show that the transition functions give an equivalence relation on the disjoint union
    $\coprod U_\alpha\times G$ and that quotienting by this relation yields a principal bundle.)
\end{exc}

